\documentclass{beamer}

\mode<presentation> {
  \usetheme{default} 
  \setbeamertemplate{navigation symbols}{} 
}

\title[NBA Draft Analysis]{NBA Draft Analysis - Which NBA Team Selects the Best Players Relative to the Average Player at their Respective Draft Position?}
\author{Luis Knufinke, Fabio Nadaia, Ralf Wyss}
\institute[Your Institute]{University of Zurich}
\date{\today}

\usepackage[table]{xcolor}

\begin{document}

\begin{frame}
  \titlepage
\end{frame}

\begin{frame}{Table of Contents}
    \begin{itemize}
        \item Introduction
        \item Player Analysis
        \item Team Analysis
        \item Conclusion
      \end{itemize}
\end{frame}

\section{Introduction}
\begin{frame}{Question}
  The NBA Draft is an annual event in the National Basketball Association (NBA)...
  \begin{itemize}
    \item Teams select eligible players to join their rosters
    \item It is a pivotal moment for the league's future
  \end{itemize}
  \vspace{10pt} 
  \textbf{Research Question:} Which NBA team selects the best players relative to the average player at their respective draft position?
\end{frame}

\section{Data}
\begin{frame}{Data}
  Information about the draft selections is from Basketball Reference and data from the NBA API...
  \begin{itemize}
    \item Period from 1996 until 2023
    \item We look at different player metrics
    \item Interactive tools to explore and compare player and team data
  \end{itemize}
\end{frame}


\section{Player Analysis}
\begin{frame}{Player Analysis}
    \frametitle{Player Analysis}
    
      \begin{itemize}
        \item Compare player metric (relative to average selection) and the draft pick
        \item Example with LeBron James and total points scored
      \end{itemize}
    
      % If your image is in a different folder or has a different name, adjust the path and filename accordingly.
      \begin{figure}
        \includegraphics[width=\textwidth]{Plots/Plot1.png}
        \caption{PTS Difference vs Draft Pick}
      \end{figure}
    
\end{frame}


\section{Player Analysis}
\begin{frame}{Player Analysis}
    \frametitle{Player Analysis}
    
      \begin{itemize}
        \item Compare player metric (raw) and the draft pick for a given draft pick
        \item Example with LeBron James and total points scored
      \end{itemize}
    
      % If your image is in a different folder or has a different name, adjust the path and filename accordingly.
      \begin{figure}
        \includegraphics[width=\textwidth]{Plots/Plot2.png}
        \caption{PTS vs Draft Pick}
      \end{figure}
    
\end{frame}

\section{Team Analysis}
\begin{frame}{Team Analysis}
    \frametitle{Team Analysis}
    
      \begin{itemize}
        \item Average performance metrics of all players selected by a team in the same draft year
        \item Example with Cleveland Cavaliers and total points scored
      \end{itemize}
    
      % If your image is in a different folder or has a different name, adjust the path and filename accordingly.
      \begin{figure}
        \includegraphics[width=\textwidth]{Plots/Plot8.png}
        \caption{PTS Difference vs Teams over Time}
      \end{figure}
    
\end{frame}

\section{Team Analysis}
\begin{frame}{Team Analysis}
    \frametitle{Team Analysis}
    
      \begin{itemize}
        \item Average differences for the selected statistic are plotted against the average draft position of each team
        \item Example with total points scored
      \end{itemize}
    
      % If your image is in a different folder or has a different name, adjust the path and filename accordingly.
      \begin{figure}
        \includegraphics[width=\textwidth]{Plots/Plot7.png}
        \caption{PTS Difference vs Average Draft Position}
      \end{figure}
    
\end{frame}



\section{Team Analysis}
\begin{frame}{Team Analysis}
    \frametitle{Team Analysis}
    
    Table showing the top 5 teams ranked by a weighted average, reflecting aggregated team performance statistics.
    
    \begin{table}
        \centering
        \caption{Top 5 Teams by Weighted Average Rank}
        \vspace{5mm} % Adjust the vertical space as needed
        \label{table:2}
        \begin{tabular}{|c|c|c|}
        \hline
        \rowcolor{gray!50} Rank & Team & Weighted Average Rank \\ % Shaded header row
        \hline
        1 & CLE & 6.00 \\
        \hline
        2 & UTA & 8.45 \\
        \hline
        3 & DEN & 8.55 \\
        \hline
        4 & MIL & 8.55 \\
        \hline
        5 & TOR & 9.09 \\
        \hline
        \end{tabular}
    \end{table}
    
    \end{frame}






\section{Conclusions}
\begin{frame}{Conclusions}
  The best NBA team at drafting is the Cleveland Cavaliers, based on the standard parameters
  \begin{itemize}
    \item Top teams relatively constant if changes weights for aggregation
    \item High ranking of teams linked to individual player selections
  \end{itemize}
\end{frame}

\section{Further Research Ideas}
\begin{frame}{Further Research Ideas}
  \begin{itemize}
    \item Investigate the relationship between team success and draft performance
    \item Conduct ablation studies
    \item Break down the analysis on a positional basis
    \item Extend the time period
    \item Add and/or change the performance metrics
  \end{itemize}
\end{frame}



\end{document}


